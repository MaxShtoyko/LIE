\documentclass{article}
\usepackage[utf8]{inputenc}
\usepackage[english,ukrainian]{babel}
\title{LIE}
\author{}
\date{November 2019}

\usepackage{amsthm}
\usepackage{amsmath}
\usepackage{amssymb}
\usepackage{natbib}
\usepackage{graphicx}

\begin{document}

Нехай  $D - R^2$  - деяка обмежена область з границею $\partial D$, що складається з дуги $\Gamma \in C^2$ і кутової точки $P$ з внутрішнім кутом $\Theta \in (0, 2\pi)$. У цій точці вектор нормалі не є неперервним, тому $\partial D$ - кусково-гладка границя з порядком гладкості 2. 

Внутрішня задача Діріхле для рівняння Лапласа полягає у знаходженні такої функції $u \in C^2(D), \quad u:D \rightarrow R$  , що задовольняє рівняння:
\begin{equation}
\Delta u = 0 \quad в \quad  D
\end{equation}
і граничну умову 
\begin{equation}
u=f \quad на \quad \partial D
\end{equation}
Розв'язок задачі (1) - (2) можна подати у вигляді потенціалу простого шару.
\begin{equation}
u(x) = \int_{\partial D} \varphi(y) \Phi(x,y) ds(y)
\end{equation}
Тут $\Phi(x,y) = \frac{1}{2\pi} ln\frac{1}{|x-y|}$ - фундаментальний розв'язок рівняння (4) , $\varphi(y)$ - невідома густина. Використовуючи подання (3) , подамо задачу (1) - (2) у наступному вигляді

	\begin{equation}
	\int_\Gamma \varphi(y) \Phi(x,y) ds(y) = f(x)
	\end{equation}
Перед тим, як чисельно розв'язати (4) за допомогою методу квадратур, ми виконаємо спеціальне нелінійне "mesh grading" перетворення.

Для того, щоб це перетворення було можливе потрібно параметризувати інтеграм (4). 

Отже, розглянемо параметризацію  $\bar{z} : [0,2\pi] \ni \partial D$. 

Вводячи кубічний поліном


де $g \geq 2$ і встановлюючи


ми вводимо "mesh grading" перетворення 


тоді  $Y \in C^{q-1}[0,2\pi]$

Тепер розглянемо нову параметризацію границі $\partial D$:
\begin{equation*}
z(s) = \tilde{z}(Y(s)), \quad 0\leq \leq 2\pi
\end{equation*}

Запишемо інтегральне рівняння у параметричній формі: 

\begin{equation}
\frac{1}{2\pi} \int_0^{2\pi} \psi(\tau) L(s,\tau)d\tau = f(s),
\end{equation}
де $s \in [0, \pi]$, функція $f(s) = f(z(s))$ і густина $\psi(s) = \varpi(z(s))  \cdot|z`(s)|$. Ядро $L(s,\tau)$  має наступну форму: 
\begin{equation*}
L(s,\tau) = -ln|z(s)- z(\tau)| \quad для s \neq \tau
\end{equation*}
У випадку коли $s=\tau$  отримуємо особливість логарифмічного типу в нулі. Перетворимо ядро $L(s,\tau)$ наступним чином 

\begin{equation*}
L(s,\tau) = -\frac{1}{2}ln \frac{4}{e^2}(cos(s) - cos(\tau))^2 + b(s,\tau)
\end{equation*}
де 

\begin{equation}
b(s,\tau) =\begin{cases}
ln\frac{2|cos(s) - cos(\tau)|}{e|z(s)-z(\tau)|} , \quad s \neq \tau \\
ln\frac{2|sin(s)|}{e|z`(s)|}, \quad \quad \quad s =\tau
\end{cases}
\end{equation}
Тут $|z`(s)| > 0$ і видно, що функція $b(s,\tau)$ не є визначена у чотирьох кутах і в центрі квадрату $[0,2\pi] \times [0,2\pi]$  і ми візьмемо це до уваги пізніше.

Звернемо увагу, що :
\begin{equation*}
\int_0^{2\pi} \varphi(\tau) ln\frac{4}{e^2} (cos(s) - cos(\tau))^2 d\tau = \\ 2 \int_0^{2\pi} \varphi(\tau)ln(\frac{4}{e}) sin^2 \frac{s-\tau}{2} d\tau, \quad s \in [0,2\pi]
\end{equation*}
 Підставимо всі ці формули в (5) 
 
 \begin{equation}
 \frac{1}{2\pi} \int_0^{2\pi} \psi(\tau) \{ -\frac{1}{2} ln\frac{4}{e} sin^2 \frac{s-\tau}{2} + b(s,\tau)\} d\tau = f(s)
 \end{equation}
\subsection{Метод квадратур}

Для чисельного розв'язування інтегрального рівняння (7) застосовуємо метод, що грунтується на тригонометричній інтерполяції з вузлами : 
\begin{equation}
s_j = \frac{j\pi}{n}, \quad 0,\dots,2n-1, \quad n \in \mathbb{N}
\end{equation}
для яких розглянемо наступні квадратурні формули

\begin{equation}
\frac{1}{2\pi} \int_0^{2\pi}f(\tau) ln(\frac{4}{e} sin^2 \frac{s-\tau}{2})d\tau \approx \sum_{j=0}^{2n-1}R_j(s)f(s_j)
\end{equation}
\begin{equation}
\frac{1}{2\pi} \int_0^{2\pi}f(\tau)d\tau \approx \frac{1}{2n}\sum_{j=0}^{2n-1}f(s_j),
\end{equation}

де $R_j(t)$ - вагові функції, що задаються у вигляді:
\begin{equation*}
R_j(t) = -\frac{1}{2n}\{1+2\sum_{m=1}^{n-1} \frac{1}{m} cos m(t - t_j) + \frac{cosn(t-t_j)}{n}\}
\end{equation*}
Після цього застосувати квадратурне формул (9)-(10), отримаємо дискретне рівняння
Для чисельного розв'язування інтегрального рівняння , ми використовуємо квадратури Аткінсона. Введемо рівномірну сітку

\begin{equation}
    s_k = \frac{k \pi}{n}, \quad k=0, \dots, 2n-1
\end{equation}

для цієї сітки вважаємо наступних два тригонометричних квадратурних правила:

\begin{equation}
\frac{1}{2\pi} \int_{0}^{2\pi} f(\sigma) \ln \left( \frac{4}{e} sin^2 \frac{s_i- \sigma}{2}\right) d\sigma \approx \sum_{k=0}^{2n-1} R_{| k- i |} f(s_k),
\end{equation}

\begin{equation}
    \frac{1}{2\pi} \int_{0}^{2\pi} f(s)ds \approx \frac{1}{2n} \sum_{k=0}^{2n} f(s_k),
\end{equation}

з вагами

\begin{equation*}
    R_j = - \frac{1}{n} \left( 1 + 2 \sum_{m=1}^{n-1}  \frac{1}{m} cos\frac{mj\pi}{n} + \frac{(-1)^j}{n^2}  \right), \quad j=0, \dots, 2n-1.
\end{equation*}

\end{document}
