\documentclass{article}
\usepackage[utf8]{inputenc}
\usepackage[english,ukrainian]{babel}
\title{LIE}
\author{}
\date{November 2019}

\usepackage{amsthm}
\usepackage{amsmath}
\usepackage{amssymb}
\usepackage{natbib}
\usepackage{graphicx}

\begin{document}

Для чисельного розв'язування інтегрального рівняння , ми використовуємо квадратури Аткінсона. Введемо рівномірну сітку

\begin{equation}
    s_k = \frac{k \pi}{n}, \quad k=0, \dots, 2n-1
\end{equation}

для цієї сітки вважаємо наступних два тригонометричних квадратурних правила:

\begin{equation}
\frac{1}{2\pi} \int_{0}^{2\pi} f(\sigma) \ln \left( \frac{4}{e} sin^2 \frac{s_i- \sigma}{2}\right) d\sigma \approx \sum_{k=0}^{2n-1} R_{| k- i |} f(s_k),
\end{equation}

\begin{equation}
    \frac{1}{2\pi} \int_{0}^{2\pi} f(s)ds \approx \frac{1}{2n} \sum_{k=0}^{2n} f(s_k),
\end{equation}

з вагами

\begin{equation*}
    R_j = - \frac{1}{n} \left( 1 + 2 \sum_{m=1}^{n-1}  \frac{1}{m} cos\frac{mj\pi}{n} + \frac{(-1)^j}{n^2}  \right), \quad j=0, \dots, 2n-1.
\end{equation*}

\end{document}

